\subsection{Champ Magnétique}

\subsubsection{Notion de champ}

\begin{definition}{type de champ}{}
    Un champ est une grandeur physique définie ne tout point M de l'espace et qui dépend de sa position et du temps.
    \begin{itemize}
        \item On parle de champ scalaire quand la valeur définie en tout point est un scalaire (température, pression\dots)
        \item On parle de champ vectoriel quand la valeur définie en tout point est un vecteur
    \end{itemize}
\end{definition}

\begin{definition}{caractéristique du champ}{}
    De mannière générale un champ dépend de deux variables. Dans des cas particulié on parle de:
    \begin{itemize}
        \item champ stationnaire quand il ne dépend que de la position. Il a la même valeur à tout instant.
        \item champ uniforme quand il ne dépend que du temps. Il a la même valeur en tout point.
    \end{itemize}
\end{definition}

\begin{definition}{}{}
    Une ligne de champ d'un champ vectoriel est une ligne qui est tangente au vecteur présent en chacun des points du champ.
\end{definition}

\subsubsection{Sources du champ magnétique}

\begin{proposition}{champ magnétique d'un fil}{}
    Pour un fil droit rectiligne infini parcourue par un courant $I$, le champ magnétique à une distance $r$ est donné par:
    \[ \vv{B} = \frac{\mu_{0} I}{2\pi r} \vv{U_{\theta}}\]
\end{proposition}

\begin{proposition}{}{}
    On parle de solénoide pour une bobine de $N$ spires, de longueur $L$ et rayon $R$ telle que $L >> R$.
    Dans un solénoide, le champ magnétique intérieur est constant et le champ magnétique extérieur est nul.
    On a la relation:
    \[\begin{array}{lcr}
         \vv{B} = \mu_{0}n I \vv{U_{z}} & \text{où} & n = \frac{N}{L}
    \end{array} \]
\end{proposition}

\subsection{Actions du champ magnétique}

\begin{proposition}{force de Laplace élémentaire}{}
    Soit un élément de courant, c'est-à-dire un fil conducteur de section $S$, de longueur $dl$, parcouru par le courant $i$ et plogné dans $\vv{B}$.
    L'ensemble des charges mobiles dans le conducteur est soumis à la force de Laplace élémentaire:
    \[ d\vv{F_{Lap}} = i d\vv{l} \wedge \vv{B} \]
\end{proposition}

\begin{proof}
    TODO
\end{proof}

\begin{proposition}{couple magnétique}{}
    Soit un moment magnétique $\vv{\mathscr{M}}$ placé dans le champ magnétique $\vv{B}$, alors le couple des actions du champ magnétique sur le moment magnétique est:
    \[ \vv{\Gamma} = \vv{\mathscr{M}} \wedge \vv{B} \]
\end{proposition}

\begin{proposition}{effet d'orientation}{}
    
\end{proposition}

force de laplace, rails, puissance

couple magnétique

effet d'orientation, équilibre

champ tournant, machine synchrone

\subsection{Lois de l'induction}

\begin{definition}{flux du champ magnétique}{}
    Soit un contour orienté de vecteur surface $\vv{S}$ et placé dans le champ magnétique $\vv{B}$ homogène. On définit le flux du champ magnétique à travers la surface:
    \[ \phi = \vv{B} \cdot \vv{S} \]
    On exprime $\phi$ en Wb (Weber) et il est proportionel à $B$ et $S$.
\end{definition}

\begin{proposition}{Loi de Lenz-Faraday}{}
    Soit un circuit electrique définissant une surface $\vv{S}$ et placée dans une zone de champ magnétique $\vv{B}$ uniforme.

    La variation du flux magnétique $\phi$ engendre un phénomèune d'induction, c'est à dire l'apparition d'un couratn dans le circuit que produirait un générateur fictif de force electromotrice $e$ telle que:
    \[ e = - \frac{d\phi}{dt} \]
\end{proposition}

L'induction revient donc à ajouter un générateur dans le circuit éléctrique. Ce générateur est toujours placé en convention générateur.

Cela est lié aux principe de modération: l'effet s'oppose toujours à celui qui lui donne naissance.

\begin{remark}{}{}
    Pour avoir un effect d'induction, il faut que $\phi$ varie.
\end{remark}

%flux du champ magnétique en Weber

%loi de faraday et implications

%générateur d'induction

%condition pour induction

\subsection{Circuit fixe dans un champ magnétique uniforme}

\subsubsection{Phenomène d'auto-induction}

Soit un solénoide de $N$ spires et de longueur $l$. Il est parcouru par un champ magnétique $\vv{B}$.

Soit une spire du solénoide parcourue par le courant $i$, de vecteur surface $\vv{S}$ et traversée par le champ magnétique $\vv{B}$. On a donc:
\[ \phi_1 = \vv{B} \cdot \vv{S} \]

\begin{definition}{flux propre}{}
    Le flux propre de la bobine $\phi_p$ est le flux qui traverse l'ensemble des spires de la bobine. On a:
    \[ \phi_p = N\phi_1 = NBS\]
\end{definition}

\begin{remark}{}{}
    Si en plus, il existe un champ magnétique extérieur $\vv{B_{ext}}$, on a:
    \[ \phi_{tot} = \phi_{p} + \phi_{ext} \]
\end{remark}

\begin{definition}{coefficient d'auto-inductance}{}
    $\phi_p$ est proportionel à $i$. On définit $L$ l'auto-inductance de la bobine telle que:
    \[ L = \frac{\phi_p}{i} \]
    $L$ s'exprime en Henry (H).
\end{definition}

\begin{remark}{}{}
    Dans le cas du solénoide, on a: $L = \frac{\mu_0 S N^2}{l}$
\end{remark}

\begin{proposition}{force electromotrice induite}{}
    D'après la loi de Lenz-Faraday, la force electromotrice d'induction est donc: $e(t) = -L\frac{di}{dt}$.

    Le generateur induit en convention générateur est donc équivalent à une bobine en convention recepteur.
\end{proposition}


TODO: Mutuelle inductance

w