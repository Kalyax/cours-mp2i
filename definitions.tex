\renewcommand*{\thesection}{\Roman{section}.}
\renewcommand*{\thesubsection}{\arabic{subsection}.}
\renewcommand*{\thesubsubsection}{\alph{subsubsection}.}

% Commands for size-adaptative parentheses, brackets, curly brackets, absolute value and magnitude.
\newcommand{\paren}[1]{\left(#1\right)} % (x)
\newcommand{\croch}[1]{\left[#1\right]} % [x]
\newcommand{\accol}[1]{\left\lbrace#1\right\rbrace} % {x}
\newcommand{\abs}[1]{\left\lvert#1\right\rvert} % |x|
\newcommand{\norme}[1]{\left\|#1\right\|} % ||x||
\newcommand{\floor}[1]{\left\lfloor#1\right\rfloor} % ⌊x⌋
\newcommand{\ceil}[1]{\left\lceil#1\right\rceil} % ⌈x⌉

% Commands for size-adaptative intervals and integer intervals. The commands' roots are "interv" and "interventier" and the added e or i at the end mean "excluded" and "included" respectively.
\newcommand{\intervii}[2]{\left[#1;#2\right]} % [a;b]
\newcommand{\intervee}[2]{\left]#1;#2\right[} % ]a;b[
\newcommand{\intervie}[2]{\left[#1;#2\right[} % [a;b[
\newcommand{\intervei}[2]{\left]#1;#2\right]} % ]a;b]
\newcommand{\interventierii}[2]{\left\llbracket#1;#2\right\rrbracket} % non-ASCII characters needed
\newcommand{\interventieree}[2]{\left\rrbracket#1;#2\right\llbracket} % non-ASCII characters needed
\newcommand{\interventierie}[2]{\left\llbracket#1;#2\right\llbracket} % non-ASCII characters needed
\newcommand{\interventierei}[2]{\left\rrbracket#1;#2\right\rrbracket} % non-ASCII characters needed

% Commands for usually used sets.
\newcommand{\N}{\mathbb{N}} % natural integers
\newcommand{\Ns}{\mathbb{N}^*}

\newcommand{\Z}{\mathbb{Z}} % relative integers
\newcommand{\Zp}{\mathbb{Z}_+}
\newcommand{\Zs}{\mathbb{Z}^*}
\newcommand{\Zps}{\mathbb{Z}_+^*}

\newcommand{\D}{\mathbb{D}} % decimal numbers
\newcommand{\Dp}{\mathbb{D}_+}
\newcommand{\Dm}{\mathbb{D}_-}
\newcommand{\Ds}{\mathbb{D}^*}
\newcommand{\Dps}{\mathbb{D}_+^*}
\newcommand{\Dms}{\mathbb{D}_-^*}

\newcommand{\Q}{\mathbb{Q}} % rational numbers
\newcommand{\Qp}{\mathbb{Q}_+}
\newcommand{\Qm}{\mathbb{Q}_-}
\newcommand{\Qs}{\mathbb{Q}^*}
\newcommand{\Qps}{\mathbb{Q}_+^*}
\newcommand{\Qms}{\mathbb{Q}_-^*}

\newcommand{\R}{\mathbb{R}} % real numbers
\newcommand{\Rp}{\mathbb{R}_+}
\newcommand{\Rm}{\mathbb{R}_-}
\newcommand{\Rs}{\mathbb{R}^*}
\newcommand{\Rps}{\mathbb{R}_+^*}
\newcommand{\Rms}{\mathbb{R}_-^*}
\newcommand{\Rb}{\overline{\mathbb{R}}}

\newcommand{\C}{\mathbb{C}} % complex numbers
\newcommand{\Cs}{\mathbb{C}^*}

\newcommand{\K}{\mathbb{K}}
\newcommand{\Ks}{\mathbb{K}^*}

\newcommand{\A}{\mathbb{A}}
\renewcommand{\L}[2]{\mathscr{L}\paren{#1,#2}}
\newcommand{\Lendo}[1]{\mathscr{L}\paren{#1}}

\newcommand{\prem}{\mathbb{P}}

\newcommand{\U}{\mathbb{U}} % complex numbers whose modulus is 1

\renewcommand{\P}[1]{\mathscr{P}\paren{#1}} % subsets of a set
\newcommand{\Pf}[1]{\mathscr{P}_f\paren{#1}} % finite subsets of a set
\newcommand{\F}[2]{\mathscr{F}\paren{#1,#2}} % functions from 1 to 2
\newcommand{\V}[1]{\mathscr{V}\paren{#1}} % neighborhood of a number

% Redefines \Re and \Im to print Re and Im (the same way as ln or lim) instead of fraktur R and I which don't look nice and are less readable.
\renewcommand{\Re}{\operatorname{Re}}
\renewcommand{\Im}{\operatorname{Im}}
\newcommand{\Card}{\operatorname{Card}}

% Command to print an upright e for the exponential instead of a slanted e and put the exponent.
\newcommand{\e}[1]{\mathrm{e}^{#1}}

% Command to print the imaginary i with a little space on the right. This way, the exponents don't look confusing. \i normally prints a dotless i.
\renewcommand{\i}{i\mkern1mu}

% Command for a QED black square. It automatically prints a whitespace before the square such that it looks nice.
\newcommand{\cqfd}{\text{ }\blacksquare}

% Commands with more explicit names for the best way to express divisibility (mid and nmid).
\newcommand{\divise}{\mid}
\newcommand{\notdivise}{\nmid}

% Commands that do the exact same thing but with explicit names for a complex number's conjugate and an event's negation in probability.
\newcommand{\conj}[1]{\overline{#1}}

% Command for a size-adaptative middle bar meaning "such that" (with spacing around it in order to look nice).
\newcommand{\tq}{\;\middle|\;}

% Command with an explicit name for the scalar product.
\newcommand{\scal}{\cdot}
\newcommand{\vecto}{\operatorname{_\wedge}}

% Shortcut for forcing displaystyle in inline mode.
\newcommand{\ds}{\displaystyle}

% Make the not version of implies, impliedby and iff look nice.
\newcommand{\notimp}{\centernot{\imp}}
\newcommand{\notimpr}{\centernot{\impr}}
\newcommand{\notssi}{\centernot{\ssi}}

\renewcommand{\subset}{\subseteq}
\renewcommand{\supset}{\supseteq}
\newcommand{\notsubset}{\centernot{\subset}}
\newcommand{\notsupset}{\centernot{\supset}}

% More explicit names for land (logical and) and lor (logical or).
\newcommand{\et}{\land}
\newcommand{\ou}{\lor}
\newcommand{\non}{\lnot}

\renewcommand{\to}{\longrightarrow}
\renewcommand{\mapsto}{\longmapsto}

\newcommand{\fonction}[5]{\begin{array}[t]{cccc}#1 : & #2 & \to & #3 \\ & #4 & \mapsto & #5\end{array}}
\newcommand{\fonctionlambda}[4]{\begin{array}[t]{ccc}#1 & \to & #2 \\ #3 & \mapsto & #4\end{array}}
\newcommand{\fonctionnd}[3]{\begin{array}[t]{ccc}#1 : & #2 \to & #3\end{array}}

\newtheoremstyle{theo}% 〈name〉
{3pt}% 〈Space above〉1
{3pt}% 〈Space below 〉1
{}% 〈Body font〉
{}% 〈Indent amount〉2
{\bfseries}% 〈Theorem head font〉
{\newline \bfseries}% 〈Punctuation after theorem head 〉
{.5em}% 〈Space after theorem head 〉3
{\thmname{#1}\thmnumber{ #2}\thmnote{ (#3)}}%

\newcounter{num}
\setcounter{num}{0}
\theoremstyle{theo}
\newtheorem{prop}[num]{Proposition}
\newtheorem{demo}[num]{Démonstration}
\newtheorem{theo}[num]{Théorème}
\newtheorem{definition}[num]{Définition}
\newtheorem{lem}[num]{Lemme}
\newtheorem{cor}[num]{Corollaire}
\newtheorem{ex}[num]{Exemple}
\newtheorem{rem}[num]{Remarque}
\newtheorem{cons}[num]{Conséquence}