\section{Séries numériques et vectorielles}

\subsection{Définitions}

\begin{proposition}{}{}
    Une suite converge si et seulement si sa série téléscopique associée converge.
    \[ \forall (u_n) \in \R^\N, ~~ (u_n) \text{ converge} ~~ \Longleftrightarrow ~~ \sum(u_{n+1} - u_n) \text{ converge} \]
\end{proposition}

\begin{enumerate}
    \item Convergence d'une série (somme partielle et reste partiel)
    \item opération sur les séries convergentes (c'est un $\K$ espace vectoriel)
    \item Lien convergence suites/séries (téléscopique, terme général tends vers 0)
\end{enumerate}

\subsection{Séries réelles à termes positifs}


\begin{proposition}{règle d'Alembert}{}
    Soit $(u_n)$ une suite réelle strictement positive, telle que $\frac{u_{n+1}}{u_n} \longrightarrow l$
    \begin{itemize}
        \item si $l < 1$, la série $\sum u_n$ converge
        \item si $l > 1$, la série $\sum u_n$ diverge
        \item si $l = 1^+$, la série $\sum u_n$ diverge
    \end{itemize}
\end{proposition}
\begin{enumerate}
    \item Règles d'Alembert
    \item Théorème de césaro
    \item Comparaison avec des ingéalités
    \item Comparaison avec des petit o ou grand O ou équivalent
    \item Implication sur des séries (jsp comment écrire)
    \item Comparaison série/intégrale
\end{enumerate}


\subsection{Séries absolument convergentes}
\begin{enumerate}
    \item En dimension finie, toute série absolument convergente est convergente
    \item Résultats sur les sommations dans les relations de dominations ???
    \item Produit de Cauchy
\end{enumerate}


\subsection{Séries alternées}
\begin{enumerate}
    \item CSSA
\end{enumerate} 

\subsection{Techniques randoms}
\begin{enumerate}
    \item quand on a un quotient $\frac{u_{n+1}}{u_n}$, passez au log pour faire des séries téléscopiques
    \item pour trouver un équivalent à une suite, voire sa série, étudier la série de la forme $\frac{1}{u_{n+1}^\alpha} - \frac{1}{u_{n}^\alpha}$
    \item quand on cherche l'équivalent $A_n$ d'une série à termes de signe non constant, on peut étudier la différence $u_n - A_n$
    \item equivalent suites récurantes gourdon p229
\end{enumerate}
