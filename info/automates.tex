\section{Langages réguliers}

\section{Automates}
\subsection{Automates fini déterministes}

\begin{enumerate}
    \item définition formelle
    \item table de transition
    \item notation fleche
    \item fonction de transition étendue
    \item mot reconnu / pas reconnu
    \item langage reconnaissable
    \item automates equivalents
    \item automate complet ($\Rightarrow$ ne bloque jamais)
    \item completer un automate
    \item émonder un automate (accessible / co-accessible) (automate qui est équivalent)
\end{enumerate}

\subsection{Automates fini non-deterministes}

\begin{enumerate}
    \item definition formelle
    \item notation fleche
    \item mot reconnu
    \item fonction de transition sur les parties
    \item fonction de transition sur les parties étendue
    \item automate des parties (déterminisation d'un automate)
    \item automates $\varepsilon$-transitions (déterminsiation aussi)
    \item $|P(Q)| = 2^{|Q|}$ nombre d'états de l'automate parties (tous ne sont pas accessible)
\end{enumerate}

\begin{proposition}{}{}
    Tout langage reconnu par un automate non-déterministe, avec ou sans $\varepsilon$-transitions,
    peut-être reconnu par un automate déterministe
\end{proposition}

\subsection{Propriétés sur les langages reconnaissables}
\begin{enumerate}
    \item Stabilité par complémentaire
    \item union
    \item concatenation
    \item étoile
    \item intersection
    \item automate produit
    \item donc la différence et la différence symétrique aussi
\end{enumerate}

\section{Théorème de Kleen}