\section{Hannah Arendt: contexte d'écriture des 2 essais}

Hannah Arendt est une philosophe ancrée dans le 20ème siècle.
Elle est à l'origine de deux aspects décisifs de la modérnité:

-- les images: leurs production, leurs statut, leurs place

-- la relfexion sur le totalitarisme, notamment le génocide des juifs

\begin{rem}
    Si on parle de shoah, on place ce génocide sur un pied d'éstale par rapport aux autres génocides.
    Si l'on ne veut pas faire de disctinction, on peut parler de génocide juif d'Europe.
\end{rem}

\subsection{Une vie confrontée au mensonge en politique}
Elle est née en 1906 en Allemagne et est d'origine juive.
Cette suite de situation font qu'elle va être en contact avec des mensonges politiques à grande échelle:

-- l'Allemagne Nazi: en 1933, elle est arrêté et comprends vite le danger

-- test