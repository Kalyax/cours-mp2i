\section{TIPE}

\subsection{Notes}

Nullstellensatz: (démo?)
\begin{itemize}
    \item Idéaux
    \item Algébriquement clos
    \item Bézout?
\end{itemize}
\mbox{}

Topologie de Zariski:????
\begin{itemize}
    \item Lemme de Zorn (AC)
\end{itemize}
\mbox{}

Dimension:?
\mbox{}

Projectif/Affine:?
\mbox{}


\subsection{Idéaux}

\begin{definition}[idéal d'un anneau]
    Soit $A$ un anneau. Un sous-ensemble $I \subset A$ est un idéal de A si:
    \begin{itemize}
        \item $(I, +)$ est un sous groupe de $(A, +)$
        \item $\forall a \in A, \forall b \in I, ab = ba \in I$
    \end{itemize}
\end{definition}


\subsection{Anneaux quotients}

Soit E un ensemble et $\sim$ une relation sur E.

\begin{definition}[relation d'équivalence]
    Une relation d'équivalence $\sim$ vérifie les propriétés suivantes sur E:
    \begin{itemize}
        \item $\sim$ réfléxive: $\forall x \in E, x \sim x$
        \item $\sim$ symétrique
        \item $\sim$ transitive
    \end{itemize}
\end{definition}

\begin{definition}[classe d'équivalence]
    Soit $x \in E$.
    L'ensemble $\tilde{x} = \{y \in E, x \sim y\}$ est la classe d'équivalence de $x$.
\end{definition}

\begin{definition}[partition]
    Une partition d'un ensemble E est définie par:
    \begin{itemize}
        \item $\biguplus_{i \in I} X_{i} = X$
        \item $\forall i \in I, X_{i} \neq \emptyset$
        \item $\forall i,j \in I, i \neq j \Rightarrow X_{i} \cap X_{j} \neq \emptyset$
    \end{itemize}
\end{definition}

\begin{definition}[ensemble quotient]
    TODO
\end{definition}