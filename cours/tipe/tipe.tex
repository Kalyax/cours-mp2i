\section{Notes}

Nullstellensatz: (démo?)
\begin{itemize}
    \item Idéaux
    \item Algébriquement clos
    \item Bézout?
\end{itemize}
\mbox{}

Topologie de Zariski:????
\begin{itemize}
    \item Lemme de Zorn (AC)
\end{itemize}
\mbox{}

Dimension:?
\mbox{}

Projectif/Affine:?
\mbox{}




\section{Rappels: relations d'équivalence}

Soit E un ensemble et $\sim$ une relation sur E.

\begin{definition}[relation d'équivalence]
    Une relation d'équivalence $\sim$ vérifie les propriétés suivantes sur E:
    \begin{itemize}
        \item $\sim$ réfléxive: $\forall x \in E, x \sim x$
        \item $\sim$ symétrique
        \item $\sim$ transitive
    \end{itemize}
\end{definition}

\begin{definition}[classe d'équivalence]
    Soit $x \in E$.
    L'ensemble $\tilde{x} = \{y \in E, x \sim y\}$ est la classe d'équivalence de $x$.
\end{definition}

\begin{definition}[partition]
    Une partition d'un ensemble E est définie par:
    \begin{itemize}
        \item $\biguplus_{i \in I} X_{i} = X$
        \item $\forall i \in I, X_{i} \neq \emptyset$
        \item $\forall i,j \in I, i \neq j \Rightarrow X_{i} \cap X_{j} \neq \emptyset$
    \end{itemize}
\end{definition}

\begin{lemma}
    Soient $x,y \in E$. On a:
    \[ x \sim y \Longleftrightarrow \title{x} = \title{y} \]
\end{lemma}

\begin{proof}
    tkt
\end{proof}

\begin{theorem}[parition formée par les classes d'équivalence]
    L'ensemble des classes d'équivalences sous $\sim$ forme une parition de $E$.
\end{theorem}

\begin{proof}
    
\end{proof}

\begin{definition}[ensemble quotient]
    TODOf
\end{definition}

<application canonique>


\section{Anneaux}

\subsection{Remarques}

\begin{definition}[anneau quotient]
    Soient $A$ un anneau et $I$ un idéal bilatère (idéal à gauche et à droite) de $A$.
    On définit la relation d'équivalence $\mathscr{R}$ suivante:

    \[ \forall x, y \in A, x \mathscr{R} y \Longleftrightarrow x - y \in I \]

    On dit aussi alors que $x$ et $y$ sont congrus modulo $I$: $x \equiv y \mod I$

    On peut munir l'ensemble quotient $A/I$ (càd l'ensemble des classes d'équivalence sur A) des lois induites par I:
    \[ \begin{array}{lcr}
        \fonction{+}{t}{a}{e}{f} & \text{et} & \fonction{\cdot}{t}{a}{e}{f}
    \end{array} \]

    $A/I$ est muni d'une structure d'anneau.

\end{definition}

\subsection{Idéaux}

\begin{definition}[idéal d'un anneau]
    Soit $A$ un anneau. Un sous-ensemble $I \subset A$ est un idéal de A si:
    \begin{itemize}
        \item $(I, +)$ est un sous groupe de $(A, +)$
        \item $\forall a \in A, \forall b \in I, ab = ba \in I$
    \end{itemize}
\end{definition}

<lien avec les noyeaux de morphismes etc>

\begin{definition}[idéal premier]
    Soit $A$ un anneau, $I$ un idéal de $A$, $I$ est premier si et seulement si l'anneau $A/I$ est intègre.
    Cela revient au même d'imposer:
    \begin{itemize}
        \item $A \neq I$
        \item $\forall a, b \in A, ab \in I \Longrightarrow a \in I$ ou $ b \in I$
    \end{itemize}
\end{definition}

\begin{definition}[idéal maximal]
    Un idéal $I$ de $A$ est dit maximal si $I \neq A$ et si pour tout idéal $J$ de $A$ tel que $I \subseteq J$ et $J \neq A$, on a $J = I$.
    ($I$ est l'élément maximal pour l'inclusion)
\end{definition}

\begin{proposition}
    Soit $I$ un idéal de $A$. On a donc:
    \begin{center}
        $I$ maximal $\Longleftrightarrow$ $A/I$ est un corps $\Longrightarrow$ $A/I$ intègre $\Longleftrightarrow$ $I$ premier
    \end{center}
\end{proposition}