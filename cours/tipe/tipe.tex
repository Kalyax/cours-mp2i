\section{Notes}

Nullstellensatz: (démo?)
\begin{itemize}
    \item Idéaux
    \item Algébriquement clos
    \item Bézout?
\end{itemize}
\mbox{}

Topologie de Zariski:????
\begin{itemize}
    \item Lemme de Zorn (AC)
\end{itemize}
\mbox{}

Dimension:?
\mbox{}

Projectif/Affine:?
\mbox{}




\section{Rappels: relations d'équivalence}

Soit E un ensemble et $\sim$ une relation sur E.

\begin{definition}{relation d'équivalence}{}
    Une relation d'équivalence $\sim$ vérifie les propriétés suivantes sur E:
    \begin{itemize}
        \item $\sim$ réfléxive: $\forall x \in E, x \sim x$
        \item $\sim$ symétrique
        \item $\sim$ transitive
    \end{itemize}
\end{definition}

\begin{definition}{classe d'équivalence}{}
    Soit $x \in E$.
    L'ensemble $\tilde{x} = \{y \in E, x \sim y\}$ est la classe d'équivalence de $x$.
\end{definition}

\begin{definition}{partition}{}
    Une partition d'un ensemble E est définie par:
    \begin{itemize}
        \item $\biguplus_{i \in I} X_{i} = X$
        \item $\forall i \in I, X_{i} \neq \emptyset$
        \item $\forall i,j \in I, i \neq j \Rightarrow X_{i} \cap X_{j} \neq \emptyset$
    \end{itemize}
\end{definition}

\begin{lemma}{}{}
    Soient $x,y \in E$. On a:
    \[ x \sim y \Longleftrightarrow \title{x} = \title{y} \]
\end{lemma}

\begin{proof}
    tkt
\end{proof}

\begin{theorem}{parition formée par les classes d'équivalence}{}
    L'ensemble des classes d'équivalences sous $\sim$ forme une parition de $E$.
\end{theorem}

\begin{proof}
    
\end{proof}

\begin{definition}{ensemble quotient}{}
    TODOf
\end{definition}

<application canonique>


\section{Anneaux}

\subsection{Remarques}

\begin{definition}{anneau quotient}{}
    Soient $A$ un anneau et $I$ un idéal bilatère (idéal à gauche et à droite) de $A$.
    On définit la relation d'équivalence $\mathscr{R}$ suivante:

    \[ \forall x, y \in A, x \mathscr{R} y \Longleftrightarrow x - y \in I \]

    On dit aussi alors que $x$ et $y$ sont congrus modulo $I$: $x \equiv y \mod I$

    On peut munir l'ensemble quotient $A/I$ (càd l'ensemble des classes d'équivalence sur A) des lois induites par I:
    \[ \begin{array}{lcr}
        \fonction{+}{t}{a}{e}{f} & \text{et} & \fonction{\cdot}{t}{a}{e}{f}
    \end{array} \]

    $A/I$ est muni d'une structure d'anneau.

\end{definition}

\subsection{Idéaux}

\begin{definition}{idéal d'un anneau}{}
    Soit $A$ un anneau. Un sous-ensemble $I \subset A$ est un idéal de A si:
    \begin{itemize}
        \item $(I, +)$ est un sous groupe de $(A, +)$
        \item $\forall a \in A, \forall b \in I, ab = ba \in I$
    \end{itemize}
\end{definition}

<lien avec les noyeaux de morphismes etc>

\begin{definition}{idéal premier}{}
    Soit $A$ un anneau, $I$ un idéal de $A$, $I$ est premier si et seulement si l'anneau $A/I$ est intègre.
    Cela revient au même d'imposer:
    \begin{itemize}
        \item $A \neq I$
        \item $\forall a, b \in A, ab \in I \Longrightarrow a \in I$ ou $ b \in I$
    \end{itemize}
\end{definition}

\begin{definition}{idéal maximal}{}
    Un idéal $I$ de $A$ est dit maximal si $I \neq A$ et si pour tout idéal $J$ de $A$ tel que $I \subseteq J$ et $J \neq A$, on a $J = I$.
    ($I$ est l'élément maximal pour l'inclusion)
\end{definition}

\begin{proposition}{}{}
    Soit $I$ un idéal de $A$. On a donc:
    \begin{center}
        $I$ maximal $\Longleftrightarrow$ $A/I$ est un corps $\Longrightarrow$ $A/I$ intègre $\Longleftrightarrow$ $I$ premier
    \end{center}
\end{proposition}


\section{Propriétés des anneaux}

\begin{definition}{éléments associés}{}
    Soit $A$ un anneau \underline{intègre}.
    Deux éléments $a$ et $b$ de $A$ sont dits associés si $a$ divise $b$ et si $b$ divise $a$.
\end{definition}

Par exemple, si on se place dans $\K[X]$, deux polynomes associés sont égaux si et seulement si ils sont unitaire.

\subsection{Notions d'idéaux}

\subsection{Types d'anneaux}

\subsubsection{Anneaux noethériens}

On rappelle qu'on idéal $I$ d'un anneau $A$ est dit de type fini s'il est engendré par un nombre fini d'éléments.

\begin{definition}{anneau noethérien}{}
    Un anneau noethérien est un anneau qui vérifie l'une des trois propriété équivalentes suivantes:
    \begin{enumerate}[(1)]
        \item tout idéal de $A$ est de type fini
        \item toute suite croissante $(I_n)_n$ d'idéaux de $A$ est stationnaire
        \item tout ensemble non vide d'idéaux de $A$ a un élément maximal pour l'inclusion
    \end{enumerate}
\end{definition}

\begin{proof} $\space$ \newline
    $(1) \Rightarrow (2)$:
    On défini une suite $(I_n)_n$ croissante et on pose $I = {\displaystyle \prod_{n \in \N}} I_n$.
    Alors il existe $N \in \N$ tel que $I \subset I_N$. 
    On a par définition de $I$: $I_N \subset I$.
    Donc $I = I_n$ 
    \newline
    $(2) \Rightarrow (3)$: TODO
    \newline
    $(3) \Rightarrow (1)$: Pas compris

\end{proof}

\begin{theorem}{Hilbert}{}
    Si $A$ est noethérien, $A[X]$ est noethérien.
\end{theorem}

\begin{corollary}{}{}
    Si $A$ est noethérien, $A[X_1,\dots,X_n]$ est noethérien.
\end{corollary}

\subsubsection{Anneaux factoriels}

La notion d'anneau factoriel généralsie la propriété de décomposition unique en facteurs premiers dans $\Z$.
Il faut noter que toutes les propriétés de $\Z$ ne s'y applique pas forcément. 

\begin{definition}{}{}
    Soit $A$ un anneau. L'anneau $A$ est factoriel s'il vérifie ces trois propriétés:
    \begin{enumerate}[(1)]
        \item $A$ est intégre (il n'a pas de diviseur de zéro)
        \item tout élément $a$ non nul de $A$ s'écrit $a = u p_1 \dots p_r$ avec $u \in A^\times$
              et $p_1, \dots, p_r$ irréductible dans $A$
        \item cette décomposition est unique, à permutation près et à des inversibles près:
              si $a = u p_1 \dots p_r = v q_1 \dots q_s$, alors $r = s$ 
              et il existe $\sigma \in \mathscr{S}_r$ tel que $p_i$ et $q_{\sigma(i)}$ soient associé
    \end{enumerate}
\end{definition}

\subsubsection{Anneaux intégralement clos}