\section{Notes}

Nullstellensatz: (démo?)
\begin{itemize}
    \item Idéaux
    \item Algébriquement clos
    \item Bézout?
\end{itemize}
\mbox{}

Topologie de Zariski:????
\begin{itemize}
    \item Lemme de Zorn (AC)
\end{itemize}
\mbox{}

Dimension:?
\mbox{}

Projectif/Affine:?
\mbox{}





\section{Ensemble quotient}

\begin{definition}{classe d'équivalence}{}
    Soit $R$ une relation d'équivalence sur un ensemble $E$.
    Soit $x \in E$, on considère la partie $\tilde{x}$ de $E$ définie par:
    \[ y \in \tilde{x} \Leftrightarrow x R y \]

    \noindent
    C'est la classe d'équivalence de $x$
\end{definition}

C'est l'ensemble des $y$ équivalent à $x$. Cette partie est non vide car $x \in \tilde{x}$.

\begin{definition}{partition}{}
    Une partition d'un ensemble E est définie par:
    \begin{enumerate}[(a)]
        \item l'union des classes d'équivalences donne $\displaystyle \biguplus_{x \in E} \tilde{x} = E$
        \item $\forall x \in E, \tilde{x} \neq \emptyset$
        \item $\forall x, y \in E, x \neq y \Rightarrow \tilde{x} \cap \tilde{y} \neq \emptyset$
    \end{enumerate}
\end{definition}

\begin{lemma}{}{}
    Soient $x, y \in E$. On a:
    \[ x \sim y \Longleftrightarrow \tilde{x} = \tilde{y} \]
\end{lemma}

On déduit que deux classes distinctes sont disjointes.

\begin{theorem}{parition formée par les classes d'équivalence}{}
    L'ensemble des classes d'équivalences sous $\sim$ forme une parition de $E$.
\end{theorem}

Les différentes classes d'équivalence des éléments de $E$ sont des parties $E$, non vides, disjointes, dont la réunion donne E (d'après la définition d'une partition).

\begin{definition}{ensemble quotient}{}
    L'ensemble des parties de $E$ dont les éléments sont des classes d'équivalence s'appelle l'ensemble quotient de $E$ par $R$,
    noté $E/R$
\end{definition}

\begin{proposition}{application canonique}{}
    Si $R$ est une relation d'équivalence, l'application \fonctionnd{\pi}{E}{E/R}
    associe un élément $x$ de $E$ à sa classe d'équivalence.

    Elle est surjective car chaque classe d'équivalence $F$ est non vide, tout élément de $F$ est envoyé par
    $\pi$ sur $F$ (donc on a: $\forall x \in F, \pi(x) = F$)
\end{proposition}








\section{Propriétés des anneaux}

\subsection{Définitions}

\begin{definition}{éléments associés}{}
    Soit $A$ un anneau \underline{intègre}.
    Deux éléments $a$ et $b$ de $A$ sont dits associés si $a$ divise $b$ et si $b$ divise $a$.
\end{definition}

Par exemple, si on se place dans $\K[X]$, deux polynomes associés sont égaux s'ils sont unitaire.

<anneaux principaux>


\subsection{Idéaux et anneau quotient}

\subsubsection{Définitions}

\begin{definition}{idéal d'un anneau}{}
    Soit $A$ un anneau. Un sous-ensemble $I \subset A$ est un idéal de A si:
    \begin{enumerate}[(a)]
        \item $(I, +)$ est un sous groupe de $(A, +)$
        \item $\forall a \in A, \forall b \in I, ab = ba \in I$
    \end{enumerate}
\end{definition}

\begin{definition}{idéal premier}{}
    Soit $A$ un anneau, $I$ un idéal de $A$, $I$ est premier si et seulement si l'anneau $A/I$ est intègre.
    Cela revient au même d'imposer:
    \begin{itemize}
        \item $A \neq I$
        \item $\forall a, b \in A, ab \in I \Longrightarrow a \in I$ ou $ b \in I$
    \end{itemize}
\end{definition}

\begin{definition}{idéal maximal}{}
    Un idéal $I$ de $A$ est dit maximal si $I \neq A$ et si pour tout idéal $J$ de $A$ tel que $I \subseteq J$ et $J \neq A$, on a $J = I$.
    ($I$ est l'élément maximal pour l'inclusion)
\end{definition}



\subsubsection{Propositions}

\begin{proposition}{}{}
    Soit $I$ un idéal de $A$. On a donc:
    \begin{center}
        $I$ maximal $\Longleftrightarrow$ $A/I$ est un corps $\Longrightarrow$ $A/I$ intègre $\Longleftrightarrow$ $I$ premier
    \end{center}
\end{proposition}

\begin{theorem}{lien idéaux et morphisme d'anneaux}{}
    Une partie $I$ d'un anneau $A$ est un idéal bilatère si et seulement si $I$
    est le noyau d'un morphisme d'anneaux.
\end{theorem}

\begin{proof}
    TODO
\end{proof}

\begin{proposition}{anneau quotient}{}
    Soit $I$ un idéal bilatère d'un anneau $A$. La relation d'équivalence $R$ définie par:
    \[ \forall x, y \in A, ~ x \mathscr{R} y \Longleftrightarrow x - y \in I \]
    est compatible avec la structure d'anneau de $A$ et l'ensemble quotient $A/R$
    aussi noté $A/I$ est muni d'une structure d'anneau.
    \newline

    On peut munir l'ensemble quotient $A/I$ (càd l'ensemble des classes d'équivalence sur A) des lois induites par I:
    \[ \begin{array}{lcr}
        \fonction{+}{t}{a}{e}{f} & \text{et} & \fonction{\cdot}{t}{a}{e}{f}
    \end{array} \]
\end{proposition}

<idéaux d'un anneau qotient>

<image d'un idéal est un idéal par un morphisme?>

<noyeau morphisme idéal?>

\begin{lemma}{}{}
    Soit $A$ un anneau, $A$ est un corps si et seulement si on a:
    \begin{enumerate}[(1)]
        \item $A \neq \{0\}$
        \item les seuls idéaux de $A$ sont $\{0\}$ et $A$
    \end{enumerate}
\end{lemma}

\begin{proof}
    TODO
\end{proof}

\begin{theorem}{Krull}{}
    Soit $I$ un idéal de $A$, $I \neq A$, il existe un idéal maximal de $A$ contenant $I$.
\end{theorem}

\begin{proof}
    Se montre à l'aide du théorème de Zorn, à voir.
\end{proof}

\subsection{Propriétés remaquables}

\subsubsection{Théoreme d'isomorphisme}

\begin{theorem}{théoreme d'isomorphisme}{}
    Soient $A$ et $B$ deux anneaux et $\fonctionnd{f}{A}{B}$ un morphisme d'anneau.
    On pose $I = \ker{f}$.

    Soit $J$ un idéal de $A$ contenu dans $I$ et $\fonctionnd{\pi}{A}{A/J}$ la projection canonique. Alors on a:
    \begin{enumerate}[(a)]
        \item il existe une unique morphisme $\fonctionnd{\overline{f}}{A/J}{B}$ tel que $f = \overline{f} \circ \pi$
                (on dit que $f$ se factorise par $A/J$)
        \item $\overline{f}$ est injectif si et seulement si $J = I$
        \item $\overline{f}$ est surjectif si et seulement $f$ l'est aussi
    \end{enumerate}

    En particulier on a $\Im{f} \simeq A/\ker{f}$.
\end{theorem}

\subsubsection{Opérations sur les idéaux}

\subsubsection{Algèbres}




\subsection{Types d'anneaux}

\subsubsection{Anneaux noethériens}

On rappelle qu'on idéal $I$ d'un anneau $A$ est dit de type fini s'il est engendré par un nombre fini d'éléments.

\begin{definition}{anneau noethérien}{}
    Un anneau noethérien est un anneau qui vérifie l'une des trois propriété équivalentes suivantes:
    \begin{enumerate}[(1)]
        \item tout idéal de $A$ est de type fini
        \item toute suite croissante $(I_n)_n$ d'idéaux de $A$ est stationnaire
        \item tout ensemble non vide d'idéaux de $A$ a un élément maximal pour l'inclusion
    \end{enumerate}
\end{definition}

\begin{proof} $\space$ \newline
    $(1) \Rightarrow (2)$:
    On défini une suite $(I_n)_n$ croissante et on pose $I = {\displaystyle \prod_{n \in \N}} I_n$.
    Alors il existe $N \in \N$ tel que $I \subset I_N$. 
    On a par définition de $I$: $I_N \subset I$.
    Donc $I = I_n$ 
    \newline
    $(2) \Rightarrow (3)$: TODO
    \newline
    $(3) \Rightarrow (1)$: Pas compris

\end{proof}

\begin{theorem}{Hilbert}{}
    Si $A$ est noethérien, $A[X]$ est noethérien.
\end{theorem}

\begin{corollary}{}{}
    Si $A$ est noethérien, $A[X_1,\dots,X_n]$ est noethérien.
\end{corollary}




\subsubsection{Anneaux factoriels}

La notion d'anneau factoriel généralsie la propriété de décomposition unique en facteurs premiers dans $\Z$.
Il faut noter que toutes les propriétés de $\Z$ ne s'y applique pas forcément. 

\begin{definition}{}{}
    Soit $A$ un anneau. L'anneau $A$ est factoriel s'il vérifie ces trois propriétés:
    \begin{enumerate}[(1)]
        \item $A$ est intégre (il n'a pas de diviseur de zéro)
        \item tout élément $a$ non nul de $A$ s'écrit $a = u p_1 \dots p_r$ avec $u \in A^\times$
              et $p_1, \dots, p_r$ irréductible dans $A$
        \item cette décomposition est unique, à permutation près et à des inversibles près:
              si $a = u p_1 \dots p_r = v q_1 \dots q_s$, alors $r = s$ 
              et il existe $\sigma \in \mathscr{S}_r$ tel que $p_i$ et $q_{\sigma(i)}$ soient associé
    \end{enumerate}
\end{definition}




\subsubsection{Anneaux intégralement clos}