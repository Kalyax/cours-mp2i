\section{Moment cinétique d'un point matériel}
\subsection{Moment cinétique}

On s'interesse tout d'abord à un point matériel $M$ de masse $m$ 
et animé de la vitesse $\vec{v}_{R}$ dans un référentiel $R$. 
\newline
\begin{defi}[quantité de mouvement]
    La quantité de mouvement du point $M$ est: \[\vec{p}_{R} = m \cdot \vec{v}_{R}\]
\end{defi}

\begin{defi}[moment cinétique par rapport à un point]
    Le moment cinétique du point $M$ par rapport au point $O$ est:
    \[\vec{L_{O}}(M) = \vec{OM} \wedge \vec{p}\]
    $\vec{L_{O}}(M)$ s'exprime en $kg \cdot m^{2} \cdot s^{-1}$ 
    et est orthogonal à $\vec{OM}$ et $\vec{v}$
\end{defi}

\begin{rem}
    On peut faire un changement d'origine d'un point $O$ vers un point $O'$:
    \[\vec{L_{O'}}(M) = \vec{O'O} \wedge m\vec{v} + \vec{L_{O}}(M)\]
\end{rem}

\begin{defi}[moment cinétique par rapport à un axe]
    Soit un axe $\Delta$ dirigé par un vecteur unitaire $\vec{u}$.
    \newline
    On définit le moment cinétique $L_{\Delta}(M)$ du point $M$ par rapport à l'axe
    $\Delta$ par:
    \[L_{\Delta}(M) = \vec{L_{O}}(M) \cdot \vec{u}\]
\end{defi}


\subsection{Moment d'une force}

\begin{defi}[moment d'une force par rapport à un point]
    Le moment de la force $\vec{F}$ qui s'exerce au point $M$ 
    par rapport au point $O$ est donnée par la relation:
    \[\vec{M_{O}}(\vec{F}) = \vec{OM} \wedge \vec{F}\]
    $\vec{M_{O}}(\vec{F})$ s'exprime en $N \cdot m$. 

    Cela traduit la capacité de la force $\vec{F}$ à faire tourner
    le point $M$ autour du point $O$.
    C'est toujours possible sauf si $\vec{F}$ est colinéaire à $\vec{OM}$
\end{defi}

\begin{rem}
    On peut faire un changement d'origine d'un point $O$ vers un point $O'$:
    \[\vec{M_{O'}}(\vec{F}) = \vec{O'O} \wedge \vec{F} + \vec{M_{O}}(\vec{F})\]
\end{rem}

\begin{defi}[moment d'une force par rapport à un axe]
    Soit un axe $\Delta$ dirigé par un vecteur unitaire $\vec{u}$. 
    \newline
    On définit le moment d'une force $M_{\Delta}(\vec{F})$ du point $M$ par rapport à l'axe
    $\Delta$ par:
    \[M_{\Delta}(\vec{F}) = \vec{M_{O}}(\vec{F}) \cdot \vec{u}\]
    $\vec{M_{O}}(\vec{F})$ s'exprime en $N \cdot m$. 

    Cela traduit la capacité de la force $\vec{F}$ à faire tourner
    le point $M$ autour de l'axe $\Delta$.
    C'est toujours possible sauf si $\vec{F}$ et $\vec{OM}$ sont coplanaire.
\end{defi}

\begin{rem}
    $M_{\Delta}(\vec{F})$ est indépendant du choix du point sur l'axe $\Delta$.
\end{rem}


\subsubsection*{Notion de bras de levier}
TODO


\subsection{Théorème du moment cinétique}

\begin{theo}[théorème du moment cinétique vectoriel]
Soit $O$ un point fixe du reférentiel $R$ galiléen.
\newline
Soit $M$ un point materiel du masse $m$, animé de la vitesse $\vec{v}$ et soumis
a un ensemble de forces $\sum_{i}\vec{f_{i}}$.
On a:
\[\frac{d\vec{L_{O}}(M)}{dt} = \sum_{i} \vec{M_{O}}(\vec{F_{i}})\]
\end{theo}

\begin{demo}
    On démontre le théorème du moment cinétique vectoriel:
    \[\frac{d\vec{L_{O}}(M)}{dt} = 
    \frac{d\vec{OM}}{dt} \wedge m\vec{v} 
    + \vec{OM} \wedge \frac{d(m\vec{v})}{dt}\]

    \[\frac{d\vec{L_{O}}(M)}{dt} = \vec{OM} \wedge m\vec{a}\]

    \[\frac{d\vec{L_{O}}(M)}{dt} = \vec{OM} \wedge \sum_{i}\vec{f_{i}}\]

    \[\frac{d\vec{L_{O}}(M)}{dt} = \sum_{i} (\vec{OM} \wedge \vec{f_{i}})\]

    \[\frac{d\vec{L_{O}}(M)}{dt} = \sum_{i} \vec{M_{O}}(\vec{F_{i}})\]
    
\end{demo}

\begin{theo}[théorème du moment cinétique scalaire]
    On projete le théorème du moment cinétique sur un axe dirigé
    par le vecteur unitaire $\vec{u}$:
    \[\frac{dL_{\Delta}(M)}{dt} = \sum_{i} M_{\Delta}(\vec{F_{i}})\]
\end{theo}

\begin{ex}
    On peut appliquer le théorème du moment cinétique sur un pendule simple ou 
    sur une bille dans une cuvette.
\end{ex}


\subsection{Cas des forces centrales}
\subsubsection{Définition}

\begin{defi}[force centrale]
    Une force $\vec{F}$ est dite centrale si sa droite support passe 
    en permanance par le point fixe $O$.
\end{defi}

\begin{cons}
    Le moment de la force $\vec{F}$ est donc nul:
    \[\vec{M_{O}} = \vec{OM} \wedge \vec{F} = \vec{0}\]
    $\vec{F}$ ne fait pas trourner le point $M$ autour de $O$
\end{cons}


\subsubsection{Conséquences pour le TMC}

