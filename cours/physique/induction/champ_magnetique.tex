\subsection{Champ Magnétique}

\subsubsection{Notion de champ}

\begin{definition}[type de champ]
    Un champ est une grandeur physique définie ne tout point M de l'espace et qui dépend de sa position et du temps.
    \begin{itemize}
        \item On parle de champ scalaire quand la valeur définie en tout point est un scalaire (température, pression\dots)
        \item On parle de champ vectoriel quand la valeur définie en tout point est un vecteur
    \end{itemize}
\end{definition}

\begin{definition}[caractéristique du champ]
    De mannière générale un champ dépend de deux variables. Dans des cas particulié on parle de:
    \begin{itemize}
        \item champ stationnaire quand il ne dépend que de la position. Il a la même valeur à tout instant.
        \item champ uniforme quand il ne dépend que du temps. Il a la même valeur en tout point.
    \end{itemize}
\end{definition}

\begin{definition}
    Une ligne de champ d'un champ vectoriel est une ligne qui est tangente au vecteur présent en chacun des points du champ.
\end{definition}

\subsubsection{Sources du champ magnétique}

\begin{proposition}[champ magnétique d'un fil]
    Pour un fil droit rectiligne infini parcourue par un courant $I$, le champ magnétique à une distance $r$ est donné par:
    \[ \vec{B} = \frac{\mu_{0} I}{2\pi r} \vec{U_{\theta}}\]
\end{proposition}

\begin{proposition}
    On parle de solénoide pour une bobine de $N$ spires, de longueur $L$ et rayon $R$ telle que $L >> R$.
    Dans un solénoide, le champ magnétique intérieur est constant et le champ magnétique extérieur est nul.
    On a la relation:
    \[\begin{array}{lcr}
         \vec{B} = \mu_{0}n I \vec{U_{z}} & \text{où} & n = \frac{N}{L}
    \end{array} \]
\end{proposition}

\subsection{chap 2}

force de laplace, rails, puissance

couple magnétique

effet d'orientation, équilibre

champ tournant, machine synchrone