\documentclass[12pt,a4paper,]{article}
\usepackage[utf8]{inputenc}
\usepackage[french]{babel}
\usepackage[T1]{fontenc}
\usepackage{amsmath}
\usepackage{amsfonts}
\usepackage{amssymb}
\usepackage{graphicx}
\usepackage{lmodern}
\usepackage{amsthm}
\usepackage[left=2cm,right=2cm,top=2cm,bottom=2cm]{geometry}
\usepackage{mathrsfs}
\usepackage{framed}
\usepackage{xcolor}
\usepackage{icomma}
\usepackage{esvect}
\usepackage{tcolorbox}
\usepackage[shortlabels]{enumitem}
\tcbuselibrary{skins, theorems}

\title{Résumé de TIPE: le Nullstellensatz de Hilbert}
\author{Alexandre}
\date{}

% Commands for size-adaptative parentheses, brackets, curly brackets, absolute value and magnitude.
\newcommand{\paren}[1]{\left(#1\right)} % (x)
\newcommand{\croch}[1]{\left[#1\right]} % [x]
\newcommand{\accol}[1]{\left\lbrace#1\right\rbrace} % {x}
\newcommand{\abs}[1]{\left\lvert#1\right\rvert} % |x|
\newcommand{\norme}[1]{\left\|#1\right\|} % ||x||
\newcommand{\floor}[1]{\left\lfloor#1\right\rfloor} % ⌊x⌋
\newcommand{\ceil}[1]{\left\lceil#1\right\rceil} % ⌈x⌉

% Commands for size-adaptative intervals and integer intervals. The commands' roots are "interv" and "interventier" and the added e or i at the end mean "excluded" and "included" respectively.
\newcommand{\intervii}[2]{\left[#1;#2\right]} % [a;b]
\newcommand{\intervee}[2]{\left]#1;#2\right[} % ]a;b[
\newcommand{\intervie}[2]{\left[#1;#2\right[} % [a;b[
\newcommand{\intervei}[2]{\left]#1;#2\right]} % ]a;b]
\newcommand{\interventierii}[2]{\left\llbracket#1;#2\right\rrbracket} % non-ASCII characters needed
\newcommand{\interventieree}[2]{\left\rrbracket#1;#2\right\llbracket} % non-ASCII characters needed
\newcommand{\interventierie}[2]{\left\llbracket#1;#2\right\llbracket} % non-ASCII characters needed
\newcommand{\interventierei}[2]{\left\rrbracket#1;#2\right\rrbracket} % non-ASCII characters needed

% Commands for usually used sets.
\newcommand{\N}{\mathbb{N}} % natural integers
\newcommand{\Ns}{\mathbb{N}^*}

\newcommand{\Z}{\mathbb{Z}} % relative integers
\newcommand{\Zp}{\mathbb{Z}_+}
\newcommand{\Zs}{\mathbb{Z}^*}
\newcommand{\Zps}{\mathbb{Z}_+^*}

\newcommand{\D}{\mathbb{D}} % decimal numbers
\newcommand{\Dp}{\mathbb{D}_+}
\newcommand{\Dm}{\mathbb{D}_-}
\newcommand{\Ds}{\mathbb{D}^*}
\newcommand{\Dps}{\mathbb{D}_+^*}
\newcommand{\Dms}{\mathbb{D}_-^*}

\newcommand{\Q}{\mathbb{Q}} % rational numbers
\newcommand{\Qp}{\mathbb{Q}_+}
\newcommand{\Qm}{\mathbb{Q}_-}
\newcommand{\Qs}{\mathbb{Q}^*}
\newcommand{\Qps}{\mathbb{Q}_+^*}
\newcommand{\Qms}{\mathbb{Q}_-^*}

\newcommand{\R}{\mathbb{R}} % real numbers
\newcommand{\Rp}{\mathbb{R}_+}
\newcommand{\Rm}{\mathbb{R}_-}
\newcommand{\Rs}{\mathbb{R}^*}
\newcommand{\Rps}{\mathbb{R}_+^*}
\newcommand{\Rms}{\mathbb{R}_-^*}
\newcommand{\Rb}{\overline{\mathbb{R}}}

\newcommand{\C}{\mathbb{C}} % complex numbers
\newcommand{\Cs}{\mathbb{C}^*}

\newcommand{\K}{\mathbb{K}}
\newcommand{\Ks}{\mathbb{K}^*}

\newcommand{\A}{\mathbb{A}}
\renewcommand{\L}[2]{\mathscr{L}\paren{#1,#2}}
\newcommand{\Lendo}[1]{\mathscr{L}\paren{#1}}

\newcommand{\prem}{\mathbb{P}}

\newcommand{\U}{\mathbb{U}} % complex numbers whose modulus is 1

\renewcommand{\P}[1]{\mathscr{P}\paren{#1}} % subsets of a set
\newcommand{\Pf}[1]{\mathscr{P}_f\paren{#1}} % finite subsets of a set
\newcommand{\F}[2]{\mathscr{F}\paren{#1,#2}} % functions from 1 to 2
\newcommand{\V}[1]{\mathscr{V}\paren{#1}} % neighborhood of a number

% Redefines \Re and \Im to print Re and Im (the same way as ln or lim) instead of fraktur R and I which don't look nice and are less readable.
\renewcommand{\Re}{\operatorname{Re}}
\renewcommand{\Im}{\operatorname{Im}}
\newcommand{\Card}{\operatorname{Card}}

% Command to print an upright e for the exponential instead of a slanted e and put the exponent.
\newcommand{\e}[1]{\mathrm{e}^{#1}}

% Command to print the imaginary i with a little space on the right. This way, the exponents don't look confusing. \i normally prints a dotless i.
\renewcommand{\i}{i\mkern1mu}

% Command for a QED black square. It automatically prints a whitespace before the square such that it looks nice.
\newcommand{\cqfd}{\text{ }\blacksquare}

% Commands with more explicit names for the best way to express divisibility (mid and nmid).
\newcommand{\divise}{\mid}
\newcommand{\notdivise}{\nmid}

% Commands that do the exact same thing but with explicit names for a complex number's conjugate and an event's negation in probability.
\newcommand{\conj}[1]{\overline{#1}}

% Command for a size-adaptative middle bar meaning "such that" (with spacing around it in order to look nice).
\newcommand{\tq}{\;\middle|\;}

% Command with an explicit name for the scalar product.
\newcommand{\scal}{\cdot}
\newcommand{\vecto}{\operatorname{_\wedge}}

% Shortcut for forcing displaystyle in inline mode.
\newcommand{\ds}{\displaystyle}

% Make the not version of implies, impliedby and iff look nice.
\newcommand{\notimp}{\centernot{\imp}}
\newcommand{\notimpr}{\centernot{\impr}}
\newcommand{\notssi}{\centernot{\ssi}}

\renewcommand{\subset}{\subseteq}
\renewcommand{\supset}{\supseteq}
\newcommand{\notsubset}{\centernot{\subset}}
\newcommand{\notsupset}{\centernot{\supset}}

% More explicit names for land (logical and) and lor (logical or).
\newcommand{\et}{\land}
\newcommand{\ou}{\lor}
\newcommand{\non}{\lnot}

\renewcommand{\to}{\longrightarrow}
\renewcommand{\mapsto}{\longmapsto}

\newcommand{\fonction}[5]{\begin{array}[t]{cccc}#1 : & #2 & \to & #3 \\ & #4 & \mapsto & #5\end{array}}
\newcommand{\fonctionlambda}[4]{\begin{array}[t]{ccc}#1 & \to & #2 \\ #3 & \mapsto & #4\end{array}}
\newcommand{\fonctionnd}[3]{\begin{array}[t]{ccc}#1 : & #2 \to & #3\end{array}}

%\newcommand{\produit}[1]{${\displaystyle \prod_{#1}^{#2}}$}

\definecolor{theoremColor}{rgb}{0.8,0.2,0.2}
\definecolor{lemmaColor}{rgb}{0.5,0.8,0.3}
\definecolor{definitionColor}{rgb}{0.15,0.4,0.8}
\definecolor{propositionColor}{rgb}{0.2,0.8,0.6}
\definecolor{corollaryColor}{rgb}{0,0.5,0.2}

% Enlever 'line cap=round' pour avoir un vrai rectangle
\tcbset{separator sign={},
        description delimiters parenthesis,
        label separator=-,
        th box/.style={
          blanker, coltitle=black, fonttitle=\scshape,
          borderline west={2pt}{0cm}{theoremColor, line cap=round},
          enlarge bottom by=9pt, left=0.5cm,
        },
        lem box/.style={
          blanker, coltitle=black, fonttitle=\scshape,
          borderline west={2pt}{0cm}{lemmaColor, line cap=round},
          enlarge bottom by=9pt, left=0.5cm,
        },
        proposition box/.style={
          blanker, coltitle=black, fonttitle=\scshape,
          borderline west={2pt}{0cm}{propositionColor, line cap=round},
          enlarge bottom by=9pt, left=0.5cm,
        },
        definition box/.style={
          blanker, coltitle=black, fonttitle=\scshape,
          borderline west={2pt}{0cm}{definitionColor, line cap=round},
          enlarge bottom by=9pt, left=0.5cm,
        },
        corollary box/.style={
          blanker, coltitle=black, fonttitle=\scshape,
          borderline west={2pt}{0cm}{corollaryColor, line cap=round},
          enlarge bottom by=9pt, left=0.5cm,
        },
        rare box/.style={
          blanker, coltitle=black, fonttitle=\scshape,
          enlarge bottom by=7pt, left=0.5cm,
        }
}

%\newtheorem{theorem}{Theoreme}

\newtcbtheorem[]{theorem}{Théorème}{th box}{theorem}
\newtcbtheorem[use counter from=theorem]{proposition}{Proposition}{proposition box}{proposition}
\newtcbtheorem[use counter from=theorem]{definition}{Définition}{definition box}{definition}

\newtcbtheorem[use counter from=theorem]{lemma}{Lemme}{lem box}{lemma}
\newtcbtheorem[use counter from=theorem]{corollary}{Corollaire}{corollary box}{corollary}

\newtcbtheorem[use counter from=theorem]{example}{Exemple}{rare box}{example}
\newtcbtheorem[use counter from=theorem]{conseq}{Conséquence}{rare box}{conseq}
\newtcbtheorem[use counter from=theorem]{remark}{Remarque}{rare box}{remark}

\begin{document}

\setcounter{page}{0}
\maketitle

\tableofcontents
\newpage

\section{Notes}

Nullstellensatz: (démo?)
\begin{itemize}
    \item Idéaux
    \item Algébriquement clos
    \item Bézout?
\end{itemize}
\mbox{}

Topologie de Zariski:????
\begin{itemize}
    \item Lemme de Zorn (AC)
\end{itemize}
\mbox{}

Dimension:?
\mbox{}

Projectif/Affine:?
\mbox{}

\newpage




\section{Notions d'ensemble}

\subsection{Relations d'ordre}

\begin{remark}{}{}
    Dans un ordre total, les notions d'élément maximal et de plus grand élément sont confondues (de même pour l'élément minimal et le plus petit élément)
\end{remark}

TODO: définition élément maximal et pge


\subsection{ZFC}

\begin{definition}{chaîne}{}
    Une chaîne d'un ensemble partiellement ordonée est une partie sur laquelle l'ordre est total.
\end{definition}

\begin{definition}{ensemble inductif}{}
    Soit $(E, \leqslant)$ un ensemble ordonné. On dit que $E$ est un ensemble inductif si toute partie
    totalement ordonné (càd toute chaine) admet un majorant dans $E$.
\end{definition}

On remarque alors que tout ensemble ordonné fini est inductif. 
Cependant ce n'est pas le cas d'ensemble comme $\Z, \Q, \R, \dots$,
on a bien $\N \subset \Z$ mais $\N$ n'admet aucun majorant dans $\Z$.

\begin{theorem}{lemme de Zorn}{}
    Tout ensemble non vide et inductif admet au moins un élément maximal.
\end{theorem}

\begin{definition}{bon ordre}{}
    Un ensemble $E$ est bien ordonnée si toute partie non vide de $E$ admet un plus petit élément. 
    
    Toute relation d'ordre vérifiant cette propriété sur $E$ est un bon ordre.
\end{definition}

On remarque aussi qu'un bon ordre est forcément total. 
Si on regarde une paire $\{a,b\}$ de $E$, on peut toujours comparer $a$ et $b$ car il y a un plus petit élément.

\begin{theorem}{théorème de Zermelo}{}
    Tout ensemble $E$ non vide, peut être bien ordonné.
\end{theorem}

\begin{theorem}{axiome du choix}{}
    Pour tout ensemble non vide $E$, il existe au moins une application $f$ de $\mathcal{P}(E)$ dans $E$ 
    telle que $\forall A \subset E$, avec $A \neq \varnothing$, on ait $f(A) \in A$.
\end{theorem}

Donc $f$ est une ``fonction de choix'' qui permet de choisir un élément dans une partie de $E$ non vide.

\begin{proposition}{}{}
    Le lemme de Zorn, le théorème de Zermelo et l'axiome du choix sont équivalent.
\end{proposition}

\begin{proof}
    Zermelo $\Longrightarrow$ Choix

    Soit $E$ un ensemble non vide qu'on munit d'un bon ordre grâce au théorème de Zermelo.
    Soit $A$ une partie non vide de $E$, alors elle admet un plus petit élément $a$.
    On pose $f(A) = a$ tel qu'on a définit $\fonction{f}{ \mathcal{P}(E)\backslash \{ \varnothing \} }{E}{A}{a}$.

    On peut ensuite définir $f(\varnothing)$ de manière quelconque pour avoir une fonction de choix $f$ sur $E$.
    Par exemple on prends $f(\varnothing) = f(E)$.

\end{proof}

Si on veut maintenant justifier Zorn $\Longrightarrow$ Zermelo, on doit d'abord considérer les parties de $E$ que l'on peut munir d'un bon ordre, et montrer que leur ensemble est inductif.

\begin{proposition}{}{}
    Soit un ensemble non vide $E$, on considère $\mathcal{M}$ l'ensemble des couples $(A,O_A)$ où $A$ est une partie non vide de $E$ et $O_A$ un bon ordre sur $A$.
\end{proposition}

\begin{remark}{}{}
    On a clairement $\mathcal{M} \neq \varnothing$ car si on prends une partie $A$ de cardinal fini $n$, 
    alors il existe une bijection $\fonction{\varphi}{[1;n]}{A}{n}{a_n}$ (avec $a_1, \dots, a_n$ éléments distincts de $A$).

    On peut ordonner $A$ par $O_A$ défini par $a_i O_A a_j \Longleftrightarrow i < j$: c'est un bon ordre car l'élément le plus petit d'une partie $A$ est associé à l'entier le plus petit.
\end{remark}
<TODO: remplacer les []>

\begin{proposition}{}{}
    On peut munir l'ensemble $\mathcal{M}$ de la relation d'ordre suivante:

    \[ (A, O_A) \preccurlyeq (B, O_B) \Leftrightarrow \left\{
        \begin{array}{ll}
            A \subset B \\
            O_A \text{ est la restriction de } O_B \text{ à } A^2 \\
            A \text{ est partie héréditaire de } B
        \end{array}
    \right. \]

\end{proposition}

\begin{proof}
    Exercice (cf Bernard Gaustiaux)
\end{proof}

<segment initial = partie héréditaire?>

\begin{proposition}{}{}
    L'ensemble $\mathcal{M}$, non vide, ainsi ordonné, est inductif.
\end{proposition}

\begin{proof}
    TODO
\end{proof}


<Pertinence de l'axiome du choix dans le cas d'un ensemble fini?>

<TODO probleme 2 alain troesh>

\subsection{Ensemble quotient}

\begin{definition}{classe d'équivalence}{}
    Soit $R$ une relation d'équivalence sur un ensemble $E$.
    Soit $x \in E$, on considère la partie $\tilde{x}$ de $E$ définie par:
    \[ y \in \tilde{x} \Leftrightarrow x R y \]

    \noindent
    C'est la classe d'équivalence de $x$
\end{definition}

C'est l'ensemble des $y$ équivalent à $x$. Cette partie est non vide car $x \in \tilde{x}$.

\begin{definition}{partition}{}
    Une partition d'un ensemble E est définie par:
    \begin{enumerate}[(a)]
        \item l'union des classes d'équivalences donne $\displaystyle \biguplus_{x \in E} \tilde{x} = E$
        \item $\forall x \in E, \tilde{x} \neq \emptyset$
        \item $\forall x, y \in E, x \neq y \Rightarrow \tilde{x} \cap \tilde{y} \neq \emptyset$
    \end{enumerate}
\end{definition}

\begin{lemma}{}{}
    Soient $x, y \in E$. On a:
    \[ x \sim y \Longleftrightarrow \tilde{x} = \tilde{y} \]
\end{lemma}

On déduit que deux classes distinctes sont disjointes.

\begin{theorem}{parition formée par les classes d'équivalence}{}
    L'ensemble des classes d'équivalences sous $\sim$ forme une parition de $E$.
\end{theorem}

Les différentes classes d'équivalence des éléments de $E$ sont des parties $E$, non vides, disjointes, dont la réunion donne E (d'après la définition d'une partition).

\begin{definition}{ensemble quotient}{}
    L'ensemble des parties de $E$ dont les éléments sont des classes d'équivalence s'appelle l'ensemble quotient de $E$ par $R$,
    noté $E/R$
\end{definition}

\begin{proposition}{application canonique}{}
    Si $R$ est une relation d'équivalence, l'application \fonctionnd{\pi}{E}{E/R}
    associe un élément $x$ de $E$ à sa classe d'équivalence.

    Elle est surjective car chaque classe d'équivalence $F$ est non vide, tout élément de $F$ est envoyé par
    $\pi$ sur $F$ (donc on a: $\forall x \in F, \pi(x) = F$)
\end{proposition}





\newpage


\section{Anneaux et idéaux}

\subsection{Définitions}

On parle d'algèbre commutative, c'est à dire que les anneaux qu'on concidère sont
commutatif pour la multiplication. On parelrai alors d'idéaux bilatères.

\begin{definition}{idéal d'un anneau}{}
    Soit $A$ un anneau. Un sous-ensemble $I \subset A$ est un idéal de A si:
    \begin{enumerate}[(a)]
        \item $(I, +)$ est un sous groupe de $(A, +)$
        \item $\forall a \in A, \forall b \in I, ab = ba \in I$
    \end{enumerate}
\end{definition}

\begin{proposition}{}{}
    L'idéal engendré par une partie $S$ de $A$ correspond à l'intersection de tous les
    idéaux de $A$ contenant $S$.
\end{proposition}

    
Si $I$ et $J$ sont des idéaux, l'ensemble $\{i + j ~ | ~ i \in I, j \in J\}$ est un idéal,
noté $I+J$. De même pour $IJ = \{ij ~ | ~ i \in I, j \in J\}$. 
De même pour l'intersection $I \cap J$



\subsection{Anneau quotient}

\begin{definition}{anneau quotient}{}
    Soit $I$ un idéal bilatère d'un anneau $A$. La relation d'équivalence $\mathcal{R}$ définie par:
    \[ \forall x, y \in A, ~ x \mathcal{R} y \Longleftrightarrow x - y \in I \]
    est compatible avec la structure d'anneau de $A$ et l'ensemble quotient $A/\mathcal{R}$
    aussi noté $A/I$ est muni d'une structure d'anneau.
\end{definition}

\begin{example}{}{}
    \begin{itemize}
        \item $A / A = \{0\}$ car il n'y a qu'une seule unique classe d'équivalence
        \item $A / \{0\} = A$ car chaque classe d'équivalence ne possède qu'un seul élément de $A$
    \end{itemize}
\end{example}


\begin{proposition}{lien idéaux et morphisme d'anneaux}{}
    Une partie $I$ d'un anneau $A$ est un idéal si et seulement si $I$
    est le noyau d'un morphisme d'anneaux.
\end{proposition}

\begin{proof}
    Si $I$ est un idéal de $A$, on conscidère le morphisme \fonction{\varphi}{A}{A/I}{a}{a + I}. \newline
    Le noyau de $\varphi$ est égal à $I$ car si $x \in \ker(\varphi)$ alors $\varphi(x) = 0 = x + I$,
    donc $x \in I$.
    L'implication réciproque est claire.
\end{proof}


\begin{theorem}{bijection idéaux d'un anneau quotient}{}
    Il existe un bijection entre les idéaux de $A/I$ et les idéaux de $A$ contenant $I$.\newline
    Si on note $p$ la surjection canonique de $A$ dans $A/I$, 
    alors l'application $J \mapsto p^{-1}(J)$ est cette bijection 
    (où $J$ est un idéal de $A/I$).
\end{theorem}



https://www.bibmath.net/ressources/justeunexo.php?id=1368

<idéaux d'un anneau qotient>

<image d'un idéal est un idéal par un morphisme?>

<noyeau morphisme idéal?>



\subsection{Propriétés}

\subsubsection{Idéaux premiers et maximaux}

\begin{definition}{idéal premier}{}
    Soit $A$ un anneau, $I$ un idéal de $A$, $I$ est premier si et seulement si l'anneau $A/I$ est intègre.
    Cela revient au même d'imposer:
    \begin{itemize}
        \item $A \neq I$
        \item $\forall a, b \in A, ab \in I \Longrightarrow a \in I$ ou $ b \in I$
    \end{itemize}
\end{definition}

\begin{definition}{idéal maximal}{}
    Un idéal $I$ de $A$ est dit maximal si $I \neq A$ et si pour tout idéal $J$ de $A$ tel que $I \subseteq J$ et $J \neq A$, on a $J = I$.
    ($I$ est l'élément maximal pour l'inclusion)
\end{definition}

\begin{lemma}{}{}
    Soit $A$ un anneau, $A$ est un corps si et seulement si on a:
    \begin{enumerate}[(1)]
        \item $A \neq \{0\}$
        \item les seuls idéaux de $A$ sont $\{0\}$ et $A$
    \end{enumerate}
\end{lemma}

\begin{proof}
    Si on a (1) et (2), on prends $a \in A$ non nul de sorte que l'idéal $(a)$
    soit non nul. On a donc $(a) = A$. Donc $1 \in (a)$. Donc il existe $x \in A$
    tel que $ax = 1$. Donc $a$ inversible. Donc $A$ corps. \newline
    Reciproquement, si $A$ est un corps et $I$ un idéal non nul. Alors on a $a^{-1}a = 1 \in I$.
    Donc $I = A$ (car $I$ possède l'unité de $A$).
\end{proof}

\begin{proposition}{}{}
    Soit $I$ un idéal de $A$. On a donc:
    \begin{center}
        $I$ maximal $\Longleftrightarrow$ $A/I$ est un corps $\Longrightarrow$ $A/I$ intègre $\Longleftrightarrow$ $I$ premier
    \end{center}
\end{proposition}

\begin{proof}
    TODO
\end{proof}

\subsubsection{Théorème des restes chinois}

\begin{proposition}{produit cartésien d'idéaux}{}
    Les idéaux de $A \times B$ sont de la forme $I \times J$ 
    où $I$ et $J$ sont des idéaux de $A$ et $B$ respectivement. 
\end{proposition}

\begin{proposition}{idéaux premiers entre eux}{}
    Soient $I$ et $J$ des idéaux de $A$. Ces idéaux sont premiers entre eux si $I + J = A$.
    
\end{proposition}

\subsubsection{Théorème de Krull}

\begin{theorem}{Krull}{}
    Soit $I$ un idéal de $A$, $I \neq A$, il existe un idéal maximal de $A$ contenant $I$.
\end{theorem}

\begin{proof}
    Se montre à l'aide du théorème de Zorn, à voir.
\end{proof}



\newpage

\section{Propriétés des anneaux}

\subsection{Définitions}

\begin{definition}{éléments associés}{}
    Soit $A$ un anneau \underline{intègre}.
    Deux éléments $a$ et $b$ de $A$ sont dits associés si $a$ divise $b$ et si $b$ divise $a$.
\end{definition}

Par exemple, si on se place dans $\K[X]$, deux polynomes associés sont égaux s'ils sont unitaire.

<anneaux principaux>



\subsubsection{Propositions}


\subsection{Propriétés remaquables}

\subsubsection{Théoreme d'isomorphisme}

\begin{theorem}{théoreme d'isomorphisme}{}
    Soient $A$ et $B$ deux anneaux et $\fonctionnd{f}{A}{B}$ un morphisme d'anneau.
    On pose $I = \ker{f}$.

    Soit $J$ un idéal de $A$ contenu dans $I$ et $\fonctionnd{\pi}{A}{A/J}$ la projection canonique. Alors on a:
    \begin{enumerate}[(a)]
        \item il existe une unique morphisme $\fonctionnd{\overline{f}}{A/J}{B}$ tel que $f = \overline{f} \circ \pi$
                (on dit que $f$ se factorise par $A/J$)
        \item $\overline{f}$ est injectif si et seulement si $J = I$
        \item $\overline{f}$ est surjectif si et seulement $f$ l'est aussi
    \end{enumerate}

    En particulier on a $\Im{f} \simeq A/\ker{f}$.
\end{theorem}

\subsubsection{Opérations sur les idéaux}

\subsubsection{Algèbres}




\subsection{Types d'anneaux}

\subsubsection{Anneaux noethériens}

On rappelle qu'on idéal $I$ d'un anneau $A$ est dit de type fini s'il est engendré par un nombre fini d'éléments.

\begin{definition}{anneau noethérien}{}
    Un anneau noethérien est un anneau qui vérifie l'une des trois propriété équivalentes suivantes:
    \begin{enumerate}[(1)]
        \item tout idéal de $A$ est de type fini
        \item toute suite croissante $(I_n)_n$ d'idéaux de $A$ est stationnaire
        \item tout ensemble non vide d'idéaux de $A$ a un élément maximal pour l'inclusion
    \end{enumerate}
\end{definition}

\begin{proof} $\space$ \newline
    $(1) \Rightarrow (2)$:
    On défini une suite $(I_n)_n$ croissante et on pose $I = {\displaystyle \prod_{n \in \N}} I_n$.
    Alors il existe $N \in \N$ tel que $I \subset I_N$. 
    On a par définition de $I$: $I_N \subset I$.
    Donc $I = I_n$ 
    \newline
    $(2) \Rightarrow (3)$: 
    Soit $E$ un ensemble non vide d'idéaux. On suppose par l'absurde que $E$ n'admet pas d'élément maximal.
    On peut alors construire par récurrence une suite $(I_n)_n$ qui contredit $(2)$. D'ou le resultat.
    \newline
    $(3) \Rightarrow (1)$: Pas compris

\end{proof}

\begin{theorem}{Hilbert}{}
    Si $A$ est noethérien, $A[X]$ est noethérien.
\end{theorem}

\begin{corollary}{}{}
    Si $A$ est noethérien, $A[X_1,\dots,X_n]$ est noethérien.
\end{corollary}




\subsubsection{Anneaux factoriels}

La notion d'anneau factoriel généralsie la propriété de décomposition unique en facteurs premiers dans $\Z$.
Il faut noter que toutes les propriétés de $\Z$ ne s'y applique pas forcément. 

\begin{definition}{}{}
    Soit $A$ un anneau. L'anneau $A$ est factoriel s'il vérifie ces trois propriétés:
    \begin{enumerate}[(1)]
        \item $A$ est intégre (il n'a pas de diviseur de zéro)
        \item tout élément $a$ non nul de $A$ s'écrit $a = u p_1 \dots p_r$ avec $u \in A^\times$
              et $p_1, \dots, p_r$ irréductible dans $A$
        \item cette décomposition est unique, à permutation près et à des inversibles près:
              si $a = u p_1 \dots p_r = v q_1 \dots q_s$, alors $r = s$ 
              et il existe $\sigma \in \mathscr{S}_r$ tel que $p_i$ et $q_{\sigma(i)}$ soient associé
    \end{enumerate}
\end{definition}




\subsubsection{Anneaux intégralement clos}

\begin{definition}{élément entier}{}
    Soit $B$ un anneau et $A$ un sous-anneau de $B$. On dit que $b \in B$ est entier sur $A$ s'il est racine d'un polynôme unitaire à cofficients dans $A$.
    C'est à dire:
    \[ b ~ \text{entier} \Longleftrightarrow \exists P \in A[X] ~ \text{unitaire}, ~ P(b) = 0 \]
\end{definition}

\begin{proposition}{anneau intégralement clos}{}
    Soit $A$ un anneau intègre. Il est dit intégralement clos si les seuls éléments entier sur $A$ de son corps des fractions $Fr(A)$ sont les éléments de $A$.
\end{proposition}

\begin{proposition}{}{}
    Tout anneau factoriel est intégralement clos.
\end{proposition}

\begin{proof}
    Soit $A$ un anneau factoriel, donc $A$ intègre. \newline
    Soit $x \in Fr(A)$ entier sur $A$. Alors il existe $a_0, \dots, a_{n-1} \in A$ tel que: 
    \[x^n + a_{n-1}x^{n-1} + \cdots + a_0 = 0\]
    On suppose par l'absurde que $x \notin A$. \newline
    On pose donc $x = \frac{y}{z}$ avec $y \in A$, $z \in A \backslash \{0,1\}$ et $y \wedge z = 1$.
    Donc:
    \[ z^n P(\frac{y}{z}) = z^n \frac{y^n}{z^n} + a_{n-1} z^n \frac{y^{n-1}}{z^{n-1}} + \cdots + a_0 z^n = 0 \]
    \[ z^n P(\frac{y}{z}) = y^n + a_{n-1} z y^{n-1} + \cdots + a_0 z^n = 0\]
    \[ y^n = z (- a_{n-1} y^{n-1} - \cdots - a_0 z^n) \]
    Or $z \nmid y^n$ car ils sont premier entre eux. Contradiction. Donc $x \in A$.
\end{proof}

\begin{example}{}{}
    Soit $d \in \Z*$ un entier sans facteur carré et différent de 1. On a alors:
    \[ d \equiv 1 [4] \Longrightarrow \Z[\sqrt{d}] ~ \text{non intégralement clos} \]
    On pensera à la contraposée comme exemple d'anneau intégralement clos.
\end{example}



\newpage

\section{Corps}

\subsection{}
%\section{Polynômes}

\section{Racines des polynômes}


\newpage
%\section{Energie d'un point materiel}

\subsection{Puissance et travail d'un force}

\begin{defi}[travail d'une force]
    C'est l'énergie fournie par cette force lorsque son point d'application se déplace.
\end{defi}

\begin{defi}[energie d'un système]
    Un système possède de l'énergie s'il est capable de fournir un travail.
    On distingue deux types d'énergie:

    -- L'énergie cinétique: si un travail peut être fourni par une modification de vitesse
    
    -- L'énergie potentielle: si un travail peut être fourni par une modification de position
\end{defi}
\newpage
\section{Moment cinétique d'un point matériel}
\subsection{Moment cinétique}

On s'interesse tout d'abord à un point matériel $M$ de masse $m$ 
et animé de la vitesse $\vec{v}_{R}$ dans un référentiel $R$. 
\newline
\begin{definition}[quantité de mouvement]
    La quantité de mouvement du point $M$ est: \[\vec{p}_{R} = m \cdot \vec{v}_{R}\]
\end{definition}

\begin{definition}[moment cinétique par rapport à un point]
    Le moment cinétique du point $M$ par rapport au point $O$ est:
    \[\vec{L_{O}}(M) = \vec{OM} \wedge \vec{p}\]
    $\vec{L_{O}}(M)$ s'exprime en $kg \cdot m^{2} \cdot s^{-1}$ 
    et est orthogonal à $\vec{OM}$ et $\vec{v}$
\end{definition}

\begin{remark}
    On peut faire un changement d'origine d'un point $O$ vers un point $O'$:
    \[\vec{L_{O'}}(M) = \vec{O'O} \wedge m\vec{v} + \vec{L_{O}}(M)\]
\end{remark}

\begin{definition}[moment cinétique par rapport à un axe]
    Soit un axe $\Delta$ dirigé par un vecteur unitaire $\vec{u}$.
    \newline
    On définit le moment cinétique $L_{\Delta}(M)$ du point $M$ par rapport à l'axe
    $\Delta$ par:
    \[L_{\Delta}(M) = \vec{L_{O}}(M) \cdot \vec{u}\]
\end{definition}


\subsection{Moment d'une force}

\begin{definition}[moment d'une force par rapport à un point]
    Le moment de la force $\vec{F}$ qui s'exerce au point $M$ 
    par rapport au point $O$ est donnée par la relation:
    \[\vec{M_{O}}(\vec{F}) = \vec{OM} \wedge \vec{F}\]
    $\vec{M_{O}}(\vec{F})$ s'exprime en $N \cdot m$. 

    Cela traduit la capacité de la force $\vec{F}$ à faire tourner
    le point $M$ autour du point $O$.
    C'est toujours possible sauf si $\vec{F}$ est colinéaire à $\vec{OM}$
\end{definition}

\begin{remark}
    On peut faire un changement d'origine d'un point $O$ vers un point $O'$:
    \[\vec{M_{O'}}(\vec{F}) = \vec{O'O} \wedge \vec{F} + \vec{M_{O}}(\vec{F})\]
\end{remark}

\begin{definition}[moment d'une force par rapport à un axe]
    Soit un axe $\Delta$ dirigé par un vecteur unitaire $\vec{u}$. 
    \newline
    On définit le moment d'une force $M_{\Delta}(\vec{F})$ du point $M$ par rapport à l'axe
    $\Delta$ par:
    \[M_{\Delta}(\vec{F}) = \vec{M_{O}}(\vec{F}) \cdot \vec{u}\]
    $\vec{M_{O}}(\vec{F})$ s'exprime en $N \cdot m$. 

    Cela traduit la capacité de la force $\vec{F}$ à faire tourner
    le point $M$ autour de l'axe $\Delta$.
    C'est toujours possible sauf si $\vec{F}$ et $\vec{OM}$ sont coplanaire.
\end{definition}

\begin{remark}
    $M_{\Delta}(\vec{F})$ est indépendant du choix du point sur l'axe $\Delta$.
\end{remark}


\subsubsection*{Notion de bras de levier}
TODO


\subsection{Théorème du moment cinétique}

\begin{theorem}[théorème du moment cinétique vectoriel]
Soit $O$ un point fixe du reférentiel $R$ galiléen.
\newline
Soit $M$ un point materiel du masse $m$, animé de la vitesse $\vec{v}$ et soumis
a un ensemble de forces $\sum_{i}\vec{f_{i}}$.
On a:
\[\frac{d\vec{L_{O}}(M)}{dt} = \sum_{i} \vec{M_{O}}(\vec{F_{i}})\]
\end{theorem}

\begin{demo}
    On démontre le théorème du moment cinétique vectoriel:
    \[\frac{d\vec{L_{O}}(M)}{dt} = 
    \frac{d\vec{OM}}{dt} \wedge m\vec{v} 
    + \vec{OM} \wedge \frac{d(m\vec{v})}{dt}\]

    \[\frac{d\vec{L_{O}}(M)}{dt} = \vec{OM} \wedge m\vec{a}\]

    \[\frac{d\vec{L_{O}}(M)}{dt} = \vec{OM} \wedge \sum_{i}\vec{f_{i}}\]

    \[\frac{d\vec{L_{O}}(M)}{dt} = \sum_{i} (\vec{OM} \wedge \vec{f_{i}})\]

    \[\frac{d\vec{L_{O}}(M)}{dt} = \sum_{i} \vec{M_{O}}(\vec{F_{i}})\]
    
\end{demo}

\begin{theorem}[théorème du moment cinétique scalaire]
    On projete le théorème du moment cinétique sur un axe dirigé
    par le vecteur unitaire $\vec{u}$:
    \[\frac{dL_{\Delta}(M)}{dt} = \sum_{i} M_{\Delta}(\vec{F_{i}})\]
\end{theorem}

\begin{ex}
    On peut appliquer le théorème du moment cinétique sur un pendule simple ou 
    sur une bille dans une cuvette.
\end{ex}


\subsection{Cas des forces centrales}
\subsubsection{Définition}

\begin{definition}[force centrale]
    Une force $\vec{F}$ est dite centrale si sa droite support passe 
    en permanance par le point fixe $O$.
\end{definition}

\begin{cons}
    Le moment de la force $\vec{F}$ est donc nul:
    \[\vec{M_{O}} = \vec{OM} \wedge \vec{F} = \vec{0}\]
    $\vec{F}$ ne fait pas trourner le point $M$ autour de $O$
\end{cons}

\begin{cons}[conséquence sur le TMC]
    Soit un point $M$ soumis à un ensemble de forces centrales de resultante $\vec{F}$.
    On a:
    \[\frac{d\vec{L_{O}}(M)}{dt} = \vec{M_{O}}(\vec{F}) = \vec{0}\]
    Donc:
    \[\vec{L_{O}}(M) = \vec{const}\]

    TODO
\end{cons}


\newpage
\section{Mouvement dans un champ newtonien}


\end{document}